\documentclass[english,a4paper,11pt]{article}
\usepackage{babel}
\usepackage{ifthen}
\usepackage{amssymb}
\usepackage[ansinew]{inputenc}

% For PDF output
%\pdfcompresslevel=9
%\pdfoutput=1

\begin{document}

\title{Hava, a Purely Functional \\ Java Virtual Machine}
\author{
  Pablo J. Pedemonte \\\\
  \small{Laboratorio de Investigaci�n y Formaci�n en Inform�tica
  Avanzada}          \\
  \small{C.C. 11, La Plata, 1900. Buenos Aires, Rep�blica Argentina} \\
  \small{\texttt{ppedemon@sol.info.unlp.edu.ar}}
}
\date{}

\maketitle

\section{What is Hava}

Hava is a Java virtual machine subset implemented fully in
Haskell. It is very inefficient, due to that fact that it doesn't
deal directly with pointers or memory, nor uses unboxed data
types or any other mechanism for speeding up pure lazy functional
programs. In addition, it is implemented in a very naive fashion.
All the data structures used for its implementation have linear
time complexity. The only exception are method bytecodes, that
were represented using arrays -- otherwise, executing a method
would have a complexity proportional to the square of the
methods's length! However, in spite of its inefficiency, it is a
rather complete JVM implementation. It supports the full bytecode
set as defined in the Java Virtual Machine Specification
\cite{jspec}, except for two operations: acquiring and releasing
locks.

In our opinion, this work has two main contributions. On one
hand, it shows that it is possible to implement a bytecode
interpreter with a strong imperative flavor (at least as
suggested by its specification) with a pure functional language.
On the other hand, we think it is valuable from an academic point
of view. Our implementation is very simple (it consists of 21
Haskell modules, with an average length of 250 lines each), what
makes it possible to understand the inner details of a commercial
object oriented language implementation, such as virtual method
invocation, class loading, exception handling, etc.

\section{What is \emph{not} Hava}

Definitely, hava is not a commercial Java virtual machine. It is
slow, and lacks some fundamental features like multithreading,
JNI, and JIT compilation. Due to the lack of JNI support, the
core library (the \texttt{java.lang} package) is limited to a
very few classes, most of them simplified with respect to the
original functionality offered by Sun's implementation.

\section{Comparison with Sun's JVM}

The functionality provided by the commercial implementation issued
by Sun that can be found also in hava is depicted below:

\begin{itemize}
\item Incremental class loading
\item Full bytecode support (except \textit{monitorenter} and \textit{monitorexit})
\item Exception handling
\item Finally blocks
\item Array management
\end{itemize}

\noindent
%
The features that hava lacks compared to Sun's Java virtual
machine are:

\begin{itemize}
\item Concurrency and multithreading
\item JNI support (this is the most serious limitation, as it inhibits the
use of many classes of the Java class library)
\item Bytecode verification (this could be added in the future)
\item Garbage Collection (it has no sense in our implementation, but it could be
implemented)
\item Access to classes in jar files
\item Reflection
\end{itemize}

\noindent
%
Some features provided by hava have some differences when
compared with their Sun's counterparts. For instance, the class
loading process is a very simplified version of the original, and
the same happens with the class initialization routine.

It is interesting to note that we could support concurrency by
using the extensions provided by Haskell. This would be quite
difficult and would requiere some major changes on our
implementation. However, with care and keeping the hacker spirit
on the rise, it could be accomplished.

Moreover, we could provide JNI emulation by allowing hava to call
Haskell functions directly. This would be relatively easy to do,
because no complex marshalling/unmarshalling of parameters would
be needed, as our virtual machine is completely implemented in
Haskell. The biggest drawback is that we will need to implement
the whole Java libraries in Haskell, so this is a lot of work
anyway.

\section{How to Use It}
Hava consists of a PERL front end (called \texttt{hava}), and the
Haskell core (called \texttt{vmmain}). First of all, you should
edit the front end and change the \texttt{HAVA\_HOME} variable to
point to the root of the hava installation (In the file is
commented where, and how to do that). The front end supports the
following parameters:

\begin{enumerate}
\item \texttt{-cp|-claspath}: define a list of directories as the
classpath (they must be separated by the '\texttt{:}' character)
\item \texttt{-?|-h|-help}: print help about the program's usage
\item \texttt{-v}: show the program's version
\end{enumerate}

\noindent
%
For instance, if you have a class named \texttt{B}, in the
package \texttt{test.methods} in the \texttt{/home/hava}
directory, you should invoke it as follows:

\begin{tabbing}
\qquad\texttt{hava -cp /home/hava test.methods.B}
\end{tabbing}

\section{How to Compile It}

You need the \textit{Glasgow Haskell Compiler}, version 5.00.2 or
higher. The source files are located in the hava root directory,
as well as the makefile script. In order to recompute dependencies
between modules, you must execute the following:

\begin{tabbing}
\qquad\texttt{make}
\end{tabbing}

\noindent
%
In order to compile it, execute this:

\begin{tabbing}
\qquad\texttt{make vmmain}
\end{tabbing}

\noindent
%
There are some circular dependencies between modules. Therefore,
is \emph{not} recommended to delete the \texttt{.hi} files before
compilation. Hopefully, they are platform independent and will
work in your platform directly, or with minor modifications that
you can apply easily over those files (such as changing the
compiler version number).

\vspace{2pc}
\begin{center}
\begin{Large}More, more, more!\ldots\end{Large}
\end{center}

\begin{thebibliography}{jspec}
\bibitem[1]{jspec}T. Lindholm and F. Yellin. \textit{The Java Virtual
Machine Specification, 2nd. Edition}. Addison--Wesley, 1999.
\end{thebibliography}

\end{document}
